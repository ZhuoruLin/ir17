\documentclass{article}

\usepackage{graphicx}
\usepackage{amsfonts,amsmath,amssymb,amsthm}
\usepackage{url}
\usepackage[usenames]{color}
\usepackage[]{algorithm2e}
\usepackage{minted}
\usepackage{booktabs}
\usepackage{amssymb}
\newcommand{\figref}[1]{Figure~\ref{#1}}

\pagestyle{empty} \addtolength{\textwidth}{1.0in}
\addtolength{\textheight}{0.5in} \addtolength{\oddsidemargin}{-0.5in}
\addtolength{\evensidemargin}{-0.5in}
\newcommand{\ruleskip}{\bigskip\hrule\bigskip}
\newcommand{\nodify}[1]{{\sc #1}} \newcommand{\points}[1]{{\textbf{[#1
points]}}}

\newcommand{\bitem}{\begin{list}{$\bullet$}%
{\setlength{\itemsep}{0pt}\setlength{\topsep}{0pt}%
\setlength{\rightmargin}{0pt}}} \newcommand{\eitem}{\end{list}}

%\input{../../defs}

\newcommand{\G}{\mathcal{G}}

%\newcommand{\bE}{\mbox{\boldmath $E$}}
%\newcommand{\be}{\mbox{\boldmath $e$}}
%\newcommand{\bU}{\mbox{\boldmath $U$}}
%\newcommand{\bu}{\mbox{\boldmath $u$}}
%\newcommand{\bQ}{\mbox{\boldmath $Q$}}
%\newcommand{\bq}{\mbox{\boldmath $q$}}
%\newcommand{\bX}{\mbox{\boldmath $X$}}
%\newcommand{\bY}{\mbox{\boldmath $Y$}}
%\newcommand{\bZ}{\mbox{\boldmath $Z$}}
%\newcommand{\bx}{\mbox{\boldmath $x$}}
%\newcommand{\by}{\mbox{\boldmath $y$}}
%\newcommand{\bz}{\mbox{\boldmath $z$}}

\newcommand{\true}{\mbox{true}}
\newcommand{\Parents}{\mbox{Parents}}

\newcommand{\ww}{{\bf w}}
\newcommand{\xx}{{\bf x}}
\newcommand{\yy}{{\bf y}}
\newcommand{\real}{\ensuremath{\mathbb{R}}}


\newcommand{\eat}[1]{}

\newcommand{\CInd}[3]{({#1} \perp {#2} \mid {#3})}
\newcommand{\Ind}[2]{({#1} \perp {#2})}

\setlength{\parindent}{0pt} \setlength{\parskip}{0.5ex}
\begin{document}
\pagestyle{myheadings} \markboth{}{DS-GA-1005/CSCI-GA.2569 Problem Set
  6  Zhuoru Lin's answer }

{\LARGE
\begin{center}Inference and Representation, Fall 2017\end{center}
}

{\Large
Structured Prediction for Part-of-Speech Tagging,
MLE and MaxEnt, Max-Sum BP
}
\begin{center}
Zhuoru Lin\\
zlin@nyu.edu
\end{center}


%{\bf Selected solutions}\\
\ruleskip 
{\em Disclaimer: 
I adhered to NYU honor code in this assignment. }

\begin{enumerate}
\item \textbf{Part-of-speech tagging using SSVM}
\begin{enumerate}
\item 

\item 
\item 
\item
\end{enumerate}
\pagebreak
\item \textbf{MaxSum}
\\
We can leverage the fglib library we used before. First we define the factor graph by the following codes. Notice how we can set the node potential by defining a prior factor that contains only one nodes. The edge potential is set by the factor that contains two nodes. The result of MaxSum algorithm returns is (0,1,1,1,1,1)

\begin{minted}{python}
fg = graphs.FactorGraph()

###### Construct Graph
x1 = nodes.VNode('X1', rv.Discrete)
x2 = nodes.VNode('X2', rv.Discrete)
x3 = nodes.VNode('X3', rv.Discrete)
x4 = nodes.VNode('X4', rv.Discrete)
x5 = nodes.VNode('X5', rv.Discrete)
x6 = nodes.VNode('X6', rv.Discrete)
# Priors 
f1 = nodes.FNode('f1', rv.Discrete([2,1], x1))
f2 = nodes.FNode('f1', rv.Discrete([2,1], x1))
f3 = nodes.FNode('f1', rv.Discrete([2,1], x1))
f4 = nodes.FNode('f1', rv.Discrete([2,1], x1))
f5 = nodes.FNode('f1', rv.Discrete([2,1], x1))
f6 = nodes.FNode('f1', rv.Discrete([2,1], x1))
# Factors
f12 = nodes.FNode('f12', rv.Discrete([[1,2],[3,4]], x1, x2))
f13 = nodes.FNode('f13', rv.Discrete([[1,2],[3,4]], x1, x3))
f14 = nodes.FNode('f14', rv.Discrete([[1,2],[3,4]], x1, x4))
f25 = nodes.FNode('f25', rv.Discrete([[1,2],[3,4]], x2, x5))
f26 = nodes.FNode('f26', rv.Discrete([[1,2],[3,4]], x2, x6))
# Register nodes
fg.set_nodes([x1,x2,x3,x4,x5,x6])
fg.set_nodes([f1,f2,f3,f4,f5,f6,f12,f13,f14,f25,f26])
# Edges
fg.set_edge(f1, x1)
fg.set_edge(f2, x2)
fg.set_edge(f3, x3)
fg.set_edge(f4, x4)
fg.set_edge(f5, x5)
fg.set_edge(f6, x6)
fg.set_edge(x1, f12)
fg.set_edge(f12, x2)
fg.set_edge(x1, f13)
fg.set_edge(f13, x3)
fg.set_edge(x1, f14)
fg.set_edge(f14, x4)
fg.set_edge(x2, f25)
fg.set_edge(f25, x5)
fg.set_edge(x2, f26)
fg.set_edge(f26, x6)
\end{minted}

\pagebreak
\item \textbf{Exponential families}
\\
The log-likelihood of the data is:
\begin{equation}
\log [p(x, \theta)] = \sum_{n=1}^{L}( \langle \theta, f(x^{(n)}) \rangle - \log[Z(\theta)])
\end{equation}
The maximum likelihood estimation $\theta_{ML}$ must satisfied:
\begin{align}
\frac{\partial p}{ \partial \theta} (\theta_{ML}) &= 0 \\
\sum_{n=1}^{L}f(x^{(n)}) - L\sum_{x} \frac{\exp[<\theta_{ML}, f(x^{n})>] }{Z(\theta_{ML})} f(x^n) &= 0\\
\frac{1}{L} \sum_{n=1}^{L}f(x^{(n)})  &= \sum_{x} p(x | \theta_{ML}) f(x^{n})
\end{align}
Technically, we also need to show that this is in fact an maximum by showing the seconderivative. This is ignored here.
\pagebreak
\item \textbf{Maximun entropy distribution}
The maximum entropy distribution can be formulated as an functinal optimization problem as:
\begin{align*}
\text{arg} \max_{p} \int_x \log(p(x)) p(x) dx \\ 
\text{  with} \int_{x} p(x) dx=1\\
 \text{ and } \int p(x) f_k(x) = \alpha_k \text{  for all k}
\end{align*}
The Lagrangian is:
\begin{equation}
\mathcal{L} = \int_x \log(p(x)) p(x) dx - \sum_{k} (\theta_k \int p(x) f_k(x)  - \alpha_k) - (\lambda \int p(x)dx -1).
\end{equation}
The Lagrange multiplier states that the optimum acheives zero derivatives of Lagragian. Taking the functional derivative, we have:
\begin{equation}
\log (p(x) ) + 1 - \lambda - \sum_{k} \theta_k f_k(x) = 0
\end{equation}
Therefore we must have:
\begin{align}
p(x) &= \exp (\sum_{k} \theta_k f_k(x) + \lambda - 1)\\
&=\frac{1}{Z} \exp (\langle \theta, f(x) \rangle)
\end{align}
with $Z = \exp(1-\lambda)$
\end{enumerate}

%\bibliographystyle{acm}
%\bibliography{reference}
\end{document}